\usepackage[utf8]{inputenc} % Textkodierung: UTF-8
\usepackage[T1]{fontenc} % Zeichensatzkodierung

\usepackage[ngerman]{babel} % Deutsche Lokalisierung
\usepackage{graphicx} % Grafiken
\usepackage{calc} % Berechnungen

% Silbentrennung:
\usepackage{hyphenat}
%\tolerance 2414
%\hbadness 2414
%\emergencystretch 1.5em
%\hfuzz 0.3pt
%\widowpenalty=10000     % Hurenkinder
%\clubpenalty=10000      % Schusterjungen
%\vfuzz \hfuzz

% Euro-Symbol:
\usepackage[gen]{eurosym}
\DeclareUnicodeCharacter{20AC}{\euro{}}

% Schriftart Helvetica:
\usepackage[scaled]{helvet}
\renewcommand{\familydefault}{\sfdefault}

\usepackage{mathptmx} % skalierbare Formelschriften

\usepackage{multicol} % mehrspaltiger Text
%\usepackage{showframe} % Seitenbegrenzungen anzeigen
\usepackage{lipsum} % Blindtext

\newcommand{\PlakatFusszeileSpaltenzahl}{}

\makeatletter
\newcommand{\@minipagerestore}{\setlength{\parskip}{\medskipamount}}%
\makeatother

% Längen:
\newlength{\PlakatSeiteHoehe}
\newlength{\PlakatSeiteRand}
\newlength{\PlakatKopfzeileHoehe}
\newlength{\PlakatKopfzeileAbstand}
\newlength{\PlakatFusszeileHoehe}
\newlength{\PlakatUniversitaetLogoBreite}
\newlength{\PlakatFakultaetLogoBreite}
\newlength{\PlakatPositionLinks}
\newlength{\PlakatKopfzeilePositionUnten}
\newlength{\PlakatFakultaetslogoPositionKorrekturLinks}
\newlength{\PlakatFakultaetslogoPositionKorrekturOben}
\newlength{\PlakatFakultaetslogoAbstandRechts}
\newlength{\PlakatFusszeilePositionUnten}
\newlength{\PlakatFusszeilePositionKorrektur}
\newlength{\PlakatTitelZweiPlatzDanach}
\newlength{\PlakatTitelDreiPlatzDanach}
\newlength{\PlakatUniversitaetslogoPositionKorrekturLinks}
\newlength{\PlakatSpaltenAbstand}
\newlength{\PlakatSchriftNormalGroesse}
\newlength{\PlakatSchriftNormalZeilenabstand}
\newlength{\PlakatAufzaehlungAbstandLinks}

% Einspaltigkeit ist nur für A3 und A4 verfügbar:
\newlength{\PlakatEinspaltigPositionKorrekturOben}
\newlength{\currentbaselineskip}
\newlength{\currentparskip}
\newlength{\currentparindent}

\newenvironment{PlakatEinspaltig}{%
    \setlength{\currentbaselineskip}{\baselineskip}%
    \setlength{\currentparskip}{\parskip}%
    \setlength{\currentparindent}{\parindent}%
    \vspace*{\PlakatEinspaltigPositionKorrekturOben}%
    \begin{minipage}[t][0cm]{\textwidth}% Höhe von 0cm sorgt für das Bleiben auf der gleichen Seite
    \setlength{\baselineskip}{\currentbaselineskip}%
    \setlength{\parskip}{\currentparskip}%
    \setlength{\parindent}{\currentparindent}%
    \raggedright\empty%
}
{%
    \end{minipage}%
}

\newsavebox{\TempBox}

\newcommand{\PlakatBildBeschnitt}{0cm 0cm 0cm 0cm}

\newcommand{\PlakatBild}[3][0cm 0cm 0cm 0cm]{%
    \vspace{-\baselineskip}%
    \begin{figure}%
        \includegraphics[width=\textwidth, trim=#1, clip=true]{#2}%
        \PlakatBildUnterschrift{#3}%
    \end{figure}%
    \vspace{-\baselineskip}%
}

\newcommand{\PlakatBildGanzseitig}[3][0cm 0cm 0cm 0cm]{%
    \begin{minipage}[t][0cm]{\paperwidth-\PlakatSeiteRand-\PlakatSeiteRand}%
    \PlakatBild[#1]{#2}{#3}%
    \end{minipage}
}

% Für Beischreibung:
\newlength{\PlakatBeschreibungBeispielbildBeschnitt}
\setlength{\PlakatBeschreibungBeispielbildBeschnitt}{0cm}
\newcommand{\PlakatBeschreibungBeispielbild}{true}
\newcommand{\PlakatBeschreibungKopfzeileUndAbsender}{true}
\newcommand{\PlakatBeschreibungDruck}{true}
\newcommand{\PlakatBeschreibungKurz}{false}
\newcommand{\TRUE}{true}